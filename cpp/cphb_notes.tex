\documentclass{article}

\usepackage{parskip}

\usepackage{amsmath} % actually amsopn
\makeatletter
\DeclareRobustCommand{\var}[1]{\begingroup\newmcodes@\mathit{#1}\endgroup}
\makeatother

\usepackage{mathtools}
\DeclarePairedDelimiter\ceil{\lceil}{\rceil}
\DeclarePairedDelimiter\floor{\lfloor}{\rfloor}

\usepackage[a4paper, margin={0.5in, 0.6in}]{geometry}

\begin{document}

\section*{Math}

\(log_k(x)\) is the number of times we have to divide \(x\) by \(k\) to get to \(1\).
\[log_k(a \, b) = log_k(a) + log_k(b)\]
\[log_k(x^n) = n \cdot log_k(x)\]
\[log_k(\frac{a}{b}) = log_k(a) - log_k(b)\]
\[log_u(x) = \frac{log_k(x)}{log_k(u)}\]

The natural logarithm \(ln(x)\) of a number \(x\) is a logarithm whose base is \(e \approx 2.71828\).\\
The number of digits of an integer \(x\) in base \(b\) is \(\floor{log_b(x) + 1}\).

\section*{Time complexity}

We can use time complexity to estimate if a particular algorithm is going to be efficient enough.
\begin{center}
	Assuming time limit is 1 second:\\
	\vspace{1mm}
	\begin{tabular}{c|c}
		Input size & Required time complexity\\
		\hline
		\(n \leq 10\) & \(O(n!)\) \\
		\(n \leq 20\) & \(O(2^n)\) \\
		\(n \leq 500\) & \(O(n^3)\) \\
		\(n \leq 5000\) & \(O(n^2)\) \\
		\(n \leq 10^6\) & \(O(n\,log\,n)\) \\
		\(n \geq 10^6\) & \(O(log\,n)\) \\
	\end{tabular}
\end{center}
It is important to remember that time complexity is only an estimate.

\end{document}
